\section{Abstract}

A fejlett szabályzó algoritmusok és a feltörekvő épületgépészeti rendszerek összehangolásával az épületek komfortszintje növelhető, az üzemeltetés költségei pedig csökkenthetők.

Szimulációval vizsgálom különböző szabályzási algoritmusok előnyeit és hátrányait, 

Szakdolgozatomban egy okosotthon-rendszerbe integrálható fűtésvezérlő rendszert szimulálok korszerű fűtési rendszerek szabályozásával foglalkozom. Az épület, illetve épületrész és a fűtési rendszer modelljét épületfizikai összefüggések segítségével írom fel. A felállított modellre szabályzót tervezek, összehasonlítom a különböző felépítésű szabályzók performanciáját.

\section{Szóbajöhető módszerek a fűtési rendszerek szabályozásánál}

Hagyományos rendszerekben termosztát használata jellemző. Itt egy mérési pont van, és innen történik a kazán be- és kikapcsolása, többnyire hiszterézises szabályzással. Az ilyen központi irányítások