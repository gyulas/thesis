\chapter{Gyakorlati megvalósítás lehetőségei}


\section{Technikai feltételek}\label{chap:feasibility-tech}

A legfontosabb technikai követelmény egy arányos szelep, ami a tömegáramot automatikusan, emberi beavatkozás nélkül képes befolyásolni. Ilyen például a Herz 7990\footnote{A szelep adatlapja: \url{http://herzmediaserver.com/data/01_product_data/01_datasheets/eng/7708-7990_en.pdf}}, \textit{Csoknyai} \cite{Herz} könyvében szerepelnek a Herz cég gépészeti termékei.

A Simulinkből történő hőmérsékletméréshez nagyon sok fejlesztésre volt szükség az iContrALL központi egységén. A rendszerbe való további integrációhoz ki kell dolgozni egy jobban skálázható megoldást (több hőmérő részére), ami meglehetősen időigényes. Ráadásul a hőmérők egyelőre egy külön központi egység közbeiktatásával kommunikálnak, ami nem teszi rugalmassá a használatot.
A beavatkozó jelek kiadása hasonló megfontolásokat igényel, integrációja ennek is időigényes.

Annak tehát, hogy egy helyiségenkénti hőmérséklet-szabályozás működjön, számos előfeltétele van. A felsoroltak pedig mind szoftveres, mind hardveres fejlesztéseket, beruházásokat igényelnek.

\section{Piaci lehetőségek}

A félévben lehetőségem volt az épületgépészettel gyakorlatban is találkozni, így jobban meg tudtam ítélni a bemutatott módszer piacképességét.

Ebben részben személyes tapasztalatokat mutatok be, amelyek nem tükrözik a teljes piac helyzetét (részben akár a trendekkel ellentétesek is lehetnek). Azzal, hogy betekintést nyertem az építőipar egy szegletébe, jobban el tudtam képzelni, mi a fizikai tartalom a sok technológia mögött, melyekkel az irodalomkutatás során találkoztam. Célom az volt, hogy képet szerezzek az alapvető elképzelésekről, elvárásokról egy HVAC rendszerrel szemben.

Egy nagy hazai kivitelező cég irodáinak látogatásakor figyeltem meg egy irodai környeztet. Arra voltam kíváncsi, adottak-e már a technikai feltételek egy ilyen szabályozás üzembe helyezéséhez, a konkrét irodában például az, hogy nagyobb átépítés nélkül\footnote{Azok a cégek, melyek azért építenek fel irodaházakat, hogy azokat bérbe adják, minél univerzálisabban szeretnének tervezni. Csak a központi magot, a szerkezeteket építik fel, a belsőépítészet, a \textit{héj} a bérlő igényei szerint valósul meg. Így előfordulhat, hogy bérlőcserekor átalakítják a bérlemény kinézetét, ekkor viszont alapvető épületgépészeti rendszerekhez nem nyúlnak hozzá.}, egy kész rendszerre is használható-e egy MPC szabályozás.

A meglátogatott épületben egy BMS (Building Management System) felügyelte a HVAC rendszereket. Ennek a géptermébe nem tudtam bemenni, de megfigyeltem az irodákban, a távhőközpontnál és a légkezelő egységeknél található gépészetet. A termosztátok és a távhőszelepek Johnson Controls gyártmányúak voltak.

Az irodákban a HVAC tervezésébe nagy mértékben beleszólt a nagy belső hőterhelés, ami a zsúfolt irodában megjelenik: a tervezők radiátoros fűtés mellett döntöttek, ezeket Danfoss elektronikus szelepek vezérlik. (Ezek pontos típusát nem tudom, de valószínűleg kétállásúak, és nem folytonosan szabályozhatók.) Szobánként lettek termosztátok elhelyezve. A szellőztetésről és a hűtésről Lindab Professional klímagerendák gondoskodnak. A BMS feladata, hogy egyszerre a fűtés és a hűtés ne legyen bekapcsolva. Az egész rendszer tervezése - igaz a főépítésszel és nem az épületgépész munkatárssal beszéltem - során a költséghatékonyságra és az alacsony karbantartási költségre törekedtek.

A technikai feltételekben a látottak alapján nem áll rosszul a piac, de ahhoz, hogy egy összetett szabályozásra költsenek egy épület tervezői, garanciát kellene adni az így elérhető megtakarításokra, amiket LEED vagy WELL tanúsítványokban extra pontokkal értékelnek.

%Külső árnyékolót sem használtak: korszerű ablakok használata mellett olyan alacsony a megtakarítás, hogy a megtérülési idő nagyon magas lett volna, illetve szélre érzékeny, karbantartást igénylő elem.