\section{Modellalkotás, irodalomkutatás}

Munkámban elsősorban a különböző fűtési típusok közti különbségeket szeretném megvizsgálni. A ház modelljét először adottnak venném, az eltérést pedig a különböző fűtési módok jelentenék.
Azaz megpróbálom felírni a környezet belső hőmérsékletre való ráhatását, eztán pedig modellezem többféle fűtőtest viselkedését.

Ehhez először áttekintettem a hőátadás lehetséges formáit és forrásait. Arra jutottam, hogy ha a levegő hőmérsékletére szabályzok, akkor az abba beleszóló tényezőket veszem sorra:
\begin{itemize}[noitemsep,topsep=0pt,parsep=0pt,partopsep=0pt]
	\item konvektív hőátadás: a felszín közelében felmelegedett levegő áramlani kezd
	\item radiatív hőátadás: sugárzással kibocsátott energia a környezetbe
\end{itemize}


A levegő hőmérsékletére ezek a következőképp hatnak a leginkább:
\begin{itemize}[noitemsep,topsep=0pt,parsep=0pt,partopsep=0pt]
	\item a fűtőtestek konvektív és radiatív hőátadással is melegítik a környezetet
	\item a radiatív energiát a tárgyak, falak nyelik el, amik ezáltal felmelegszenek (mintegy kapacitásként lesz egy hőtároló tömeg, ami a fűtés kikapcsolásával fenntartja a hőmérsékletet / lassítja a hűlést)
	\item a fűtetlen falfelületek hűtik a szobát (külső hőmérséklet befolyása)
\end{itemize}

Így a kezdeti modellben azzal a feltételezéssel élek, hogy ezen kívül más hatás nem lép fel.

A modellben feltételezem, hogy a fűtőtest felületi hőmérsékletével tudunk beavatkozni. A modellben paraméter a fűtőtestek hőátadási tényezője és felülete. Zavarásként (?) hat a külső hőmérséklet értéke, amit mérni is tudunk. Kimenet a belső hőmérséklet (térben konstansnak véve azt / átlagolva a szoba levegőjére)

A modell felírásához a fűtőtest tulajdonságain kívül szükség van a szobában található levegő mennyiségére is. A zavarás hatását is fel kell írni, azaz hogy egy külső hőmérsékletváltozás hogyan jelenik meg a kimeneten. (Célszerű itt egy átviteli függvényt felírni először, szuperpozíciószerűen. A zavarás viszont nem a modell bemenetén és nem is a kimenetén hat.)

A felírandó átviteli függvények:

\begin{itemize}[noitemsep,topsep=0pt,parsep=0pt,partopsep=0pt]
	\item levegő felmelegedése konstans külső hőmérsékletet feltételezve, fűtőtest egységugrással
	\item levegő felmelegedése fűtés kikapcsolt állapota mellett, környezeti hőmérséklet ugrásával
\end{itemize}


\subsection{Radiátor modelljének felírása}

Mivel a Matlab heater modelljének teljesítmény kimenete van, fel akartam állítani egy olyan fűtőtest modellt, ami beillesztehető az eredeti légbefúvó rendszerhelyére. Ehhez megvizsgáltam a fűtési rendszer tulajdonságait:


Gouda2000 és mások alapján számolva irreális teljesítményértékeket kaptam (150kW), így tovább kutattam a szakirodalmat.

Az \textit{Épületgépészet a gyakorlatban} (5.11.6, 2. o.) egy Dunaferr radiátor tényleges hőleadását vezetik le. Ebben a hőátbocsátási tényezőt is hőmérsékletfüggőnek veszik.

Ez bővebben a \textit{Herz II - Több mint hidraulika} 4.2.4.1 (Fűtőtest lehűlése) részben is szerepel.

Pontosabban a 4.2.7.3 - Radiátorok résznél olvasható. Itt a hőveszteség adott. Esetünkben ezt a házra a Matlab számolja és jól méretezett rendszert tételezünk fel. Csupán azért kell a hőleadást jól felírni, hogy a felfutás, hőkapacitás, stb. során átadott energiát is belekalkuláljuk.

Persze ilyenkor egyedi esetekből indulok ki, de remélhetőleg ez paraméterezhetően elvezet az általános, többféle házra alkalmazható megoldáshoz.

A fűtőtestek hőleadását befolyásolja a fűtőtestek közepes hőmérsékletkülönbsége (\ref{termeszeteshk}, ahol $t_e$ és $t_v$ az előremenő és visszamenő vízhőmérsékletek és  $t_{bo}$ a helyiség belső hőmérséklete), a felülete és a hőleadási tényezője. (A 86. oldalon $\Delta t_k$, a 358.-on $\Delta t_m$ jelöléssel találkozunk. A 359. oldalon ismét változik ugyanannak a jelölése. (\ref{termeszeteshk_359}) Ezutóbbi angol jelölés szimpatikusabb.)

A hőleadás egyenlete általánosan \ref{holeadas} (Több, m. h. 358.~o.), ahol a $\dot{Q}$ [\SI{}{\watt}] a leadott hő, $\Delta t_m$ [\SI{}{\kelvin}] a közepes hőmérsékletkülönbség.




%\begin{equation} \label{termeszeteshk}
%\Delta t_k = \frac{t_e+t_v}{2} -t_{bo}
%\end{equation}

%\begin{equation} \label{termeszeteshk_359}
%\Delta t_m = \frac{t_s+t_r}{2} -t_{i}
%\end{equation}


\begin{equation} \label{holeadas}
\dot Q = k_e \cdot A_e \cdot \Delta t_m
%
%\begin{pmatrix}
%x_{1}\\
%x_{2}
%\end{pmatrix}
%+
%0 \cdot{} F
\end{equation}

\begin{equation} \label{termeszeteshk_359}
\Delta t_m = \frac{t_s+t_r}{2} -t_{i}
\end{equation}

A hőátadási tényező egy bizonyos hőmérsékleten:

\begin{equation} \label{termeszeteshk_359}
k_e = \frac{\dot{Q}}{A\cdot \Delta t_m}
\end{equation}

A hőteljesítmény hőmérsékletfüggő (361.~o.). Az $x^{1.33}$ az egyenletekben $x\cdot x^{1/3}$, ebből pedig $ x \cdot \sqrt[3]{x}$ formában jelenik meg.


Nominálisan $\Delta t_m$ = \SI{60}{\celsius}-ra adott érték a közepes hőmérsékletkülönbség függvényében változik:


\pagebreak