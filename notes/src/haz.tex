\section{Ház modellje}

\subsection{Big picture}

A modellalkotásnál igyekszek energetikai tanúsítványban szereplő adatokat felhasználni.
Figyelembe kell vennem a ház hőveszteségeit és hőtároló képességét is.

A kinyerhető adatok: a határoló elemek felszíne,

hőigény numerikusan is szerepel
A Simscape-ben hőátadási tényezőket és hőtároló tömegeket vettem fel.


\section{A felírt modell}

A schönherzes kollégiumi szoba határoló elemeit vettem fel:
\begin{table}[H]
	\centering
	
	\renewcommand{\arraystretch}{2} % to increase cell height
	\taburulecolor{gray}
	
	%\begin{tabular}{|p{0.8cm}|p{1cm}|p{1cm}|p{1cm}|p{1cm}|p{1cm}|p{1cm}|p{1cm}|}
	
	\newcolumntype{C}[1]{>{\centering\arraybackslash}p{#1}}
\newcolumntype{R}[1]{>{\raggedleft\arraybackslash}p{#1}}

\begin{tabu}{|p{3cm}|p{1cm}|p{3cm}|p{3cm}|p{3cm}|}
	%{p{1.5cm}|C{0.8cm}|C{0.8cm}|C{0.8cm}|C{0.8cm}|C{0.8cm}|C{0.8cm}|C{0.8cm}|C{0.8cm}|}
	%\multicolumn{1}{l}{}&\multicolumn{8}{l}{SDO header (első adatbyte) - master kérése}
	%\\ 		\cline{2-9}\cline{2-9}
	\hline
	felület& méret & kalorikus hőátbocsátási tényező    & hőtároló tömeg & hőkapac
	\\ \hline
	külső fal & 4.5 \si{\metre\squared} & 2 \si[per-mode=fraction]{\watt\per\metre\squared\per\kelvin} & 4.5*200kg & e.g. 4.5*200*840 \si[per-mode=fraction]{\joule\per\kelvin}
	\\ \hline
	ablak & 4 \si{\metre\squared} & 4 \si[per-mode=fraction]{\watt\per\metre\squared\per\kelvin} & 0 & 0
	\\ \hline
	belső válaszfalak & 50 \si{\metre\squared} & 7 & 50*100kg & 50*100*840	
	\\ \hline
	padló & 16 \si{\metre\squared} & 11 & 16*200kg & 169*200*840	
	\\ \hline
	mennyezet & 16 \si{\metre\squared} & ? rad / conv &  & 	
	\\ \hline
%	belső válaszfalak & 50 \si{\metre\squared} & 7 & 50*100kg & 50*100*840		
%	\\ \hline
%	11 & Internal limit active
%	\\ \hline
%	12-13 & Operation mode specific
%	\\ \hline
%	14-15 & Reserved
\end{tabu}

	
	\caption{Különböző SDO típusok felépítése - minden adat hexában értendő}
	\label{table-sdotypes}
\end{table}



A belső falakon nem vesztünk hőt, csak az ablakon ill. a külső falon.
A napsütés hőnyereségét is figyelembe vehetjük, úgy, hogy egy hőforrás a padlót melegíti.



\subsection{Fűtési rendszer és ház kapcsolata}





A fűtési rendszer és a szabályzás alapvető validálásához egyszerű házmodelleket fogok felállítani.

Szinte a legegyszerűbb ilyen tekintetben egy kollégiumi szoba modellje. Egy átlagos szobát 4 másik vesz körül, van ablaka, egy radiátora.
Erre ki kellene számítani a hőigényt, figyelembe véve azt hogy mennyi hő szökik el a külső és belső határoló felületeken keresztül.
A gyakorlati alkalmazásokban szeretnék majd az energetikai tanúsítványból kiindulni, így gyakorlatilag a szoba energetikai tanúsítását végzem el - olyan szinten, amennyire nekem szükséges.


Ashrae HVAC - 6.19 Panel H \& C. - Controls strategy

Ezért utánanéztem a jellemző szerkezeti tulajdonságoknak. A modellezés Gouda alapján történik, gyakorlatilag csomóponti egyenleteket kell felírni az alábbi hálózatra, amiben az ellenállások a rétegrendi hőátbocsátási tényező reciprokai. A hőtároló képességeket kapacitások modellezik. Ezeket az elemeket Simscape-ben implementáltam, a hőáramok így áttekinthetők és a paraméterek könnyen változtathatók.

A ház modelljének felírásakor figyelembe vettem a hőtároló elemeket. A pontos (reális) modell felállításakor ezek hőtartalmát (a hőáram integrálja egyensúlyi állapotban legyen 0, azaz egy nagyobb ciklusban a felvett és leadott hője egyenlő) az egyensúlyi állapothoz közelinek vettem.

Viszont a szabályzótervezéshez identifikálni kell, ekkor pedig a falak, ill. szoba levegőjének kezdeti állapotát (hőmérsékletét) azonosnak vettem a külső hőmérséklettel. Így ha a hőkülönbség a modell kimenő jele, akkor lineáris a rendszer: 0 bemenetre (fűtés) 0 kimenetet ad.

\subsection{Megvalósítás MATLAB-ban}




\pagebreak