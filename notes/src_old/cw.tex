\subsection{Motor bekapcsolási folyamata}\label{setting-motion-on-gold-twitter}

Bekapcsolás alatt a motorvezérlő teljesítményfokozatának be-és kikapcsolásáshoz szükséges parancsokat értjük. A folyamat során a CANopen master eszköz a motorvezérlő állapotgépét irányítja.

%Ezek az üzenetváltások az NI-CAN bus monitorral lettek rögzítve (logolva). Mellékelve megtalálható a logfile és a képernyőkép.
%\begin{scriptsize}
%    \textcolor{red}{A logolva meg zarojelbe sem kell oda! Milyen képernyő kép és milyen log file és hova van mellékelve? 
%        Az egyes üzeneteket, hogy mi a pontos tartalmuk, szerintem elég appendixben közölni. Nem rontja el a képet illetve nem zavarja meg az olvasót a belső tartalmával.
%    }
%\end{scriptsize}

%\begin{scriptsize}
%    \textcolor{red}{Mi alapján írtál ide dolgokat? Mi a sorrendjük? Írd le h mire lehet itt számítani, mert különben durván zavaros, főleg így tördelés mentesen! Ellenben tartalmilag biztos h jó.
%    }
%\end{scriptsize}




\subsubsection{Logikai táp bekapcsolása}\label{after-turn-on}

Közvetlenül akkor, amikor a motorvezérlő megkapja a logikai tápját, a \textit{Switch on disabled} állapotba kerül. Ezt a CAN buszon egy ún. heartbeat üzenet kiküldésével jelzi, ami az alábbi:

\begin{table}[H]
	\centering
	\renewcommand{\arraystretch}{2} % to increase cell height
	\taburulecolor{gray}
	
	\begin{tabu}{|p{1.3cm}|p{1cm}|p{1.2cm}|}
		\hline
		COBID & length & data
		\\ 		\hline
		0x7FF & 1 & 00
		\\		\hline
	\end{tabu}
	\caption*{Heartbeat message}
\end{table}

Mivel a CAN busz olyan, hogy a fogadó egységnek minden csomagot ACK-olni kell, ha még nem áll készen a fogadó egység és ez nem történik meg, a Twitter akkor a heartbeat-et folyamatosan fogja küldeni (az ACK akkor történik meg). A motorvezérlőt irányító RX-MUX egység a heartbeat megérkezése után tudja, hogy megkezdheti üzenetek küldését.

A belső állapotgép lekérdezése a statusword lekérésével zajlik.
%Then, you can query its status by the following SDO:



\begin{table}[H]
	\centering
	\renewcommand{\arraystretch}{2} % to increase cell height
	\taburulecolor{gray}
	
	\begin{tabu}{|p{1.3cm}|p{1cm}|p{2cm}|p{2cm}|}
		\hline
		COBID & length & data & data
		\\ 		\hline
		0x67F & 8 & 40 41 60 00 & 00 00 00 00
		\\		\hline
		0x5FF & 8 & 4B 41 60 00 & 50 02 00 00
		\\		\hline
	\end{tabu}
	\caption*{Status query by SDO}
\end{table}

\begin{formal}
	Az üzenet a \ref{table-sdotypes} szerinti felépítésű, azaz az 1. byte jelzi hogy lekérés, a 6041-es indexű objektum 0-s subindex mezőjét kérjük. A kérésben az SDO alsó 4 byteja (nibble-je) üresen marad.
	
	A válaszban 4B jelzi a 2 bytenyi hasznos adat érkezését. Az index, subindex mezők változatlanok, ezt követi a 2 bytenyi adat. Az adatok értelmezését egyértelműen biztosítja a jelzett index és subindex. (Ehhez nyilván ismerni kell az OD felépítését.)
\end{formal}


NMT reset üzenet kiadása után a PDO-k aktiválódnak. Ezek között van a gyári default TPDO1, ami a statusword-öt tartalmazza. Újabb SDO commandword kiadásával ismét állapotot válthatunk.


\begin{table}[H]
	\centering
	\renewcommand{\arraystretch}{2} % to increase cell height
	\taburulecolor{gray}
	
	\begin{tabu}{|p{1.3cm}|p{1cm}|p{4cm}|}
		\hline
		COBID & length & data
		\\ 		\hline
		0x000 & 8 & 01 00 00 00 00 00 00 00
		\\		\hline
		0x1FF & 2 & \textcolor{red}{50 02}
		\\		\hline
	\end{tabu}
	\caption*{NMT command}
\end{table}

% After the NMT, the TPDO1 reflects statusword. It refreshess on changing.
%(e.g.~battery is plugged off or a setpoint in motion is reached.)

NMT Operational állapotba lépéssel (DS301 p.67) a TPDO1 automatikusan elküldésre kerül, ha a statusword megváltozik. Ez lehet pl. a nagyfeszültségű tápforrás bekapcsolása, a motor engedélyezése vagy a \ref{table-statusword}. táblázat szerint más mező változása.

A motorvezérlő ekkor \texttt{switch on disabled} állapotban van. Az alábbi controlword beállítással \texttt{Switched ON} állapotba lépünk.
 
\begin{table}[H]
	\centering

	\renewcommand{\arraystretch}{2} % to increase cell height
	\taburulecolor{gray}
	
	\begin{tabu}{|p{1.3cm}|p{1cm}|p{2cm}|p{2cm}|}
		\hline
		COBID & length & data & data
		\\ 		\hline
		0x67F & 8 & 22 40 60 00 & 06 00 00 00		
		\\		\hline
		0x5FF & 8 & 60 40 60 00 & 00 0 00 00
		\\		\hline
		0x1FF & 2 & &
		\\		\hline
	\end{tabu}
	\caption*{Switch on}
\end{table}

\begin{formal}
	Az első üzenet legelső adatbytejának \texttt{0x22} értéke jelzi hogy az SDO paramétert állít be. A paraméter az Object Dictionary \texttt{0x6040}-es indexű, \texttt{0x00} subindexű eleme. A beállítandó érték fordított bytesorrendben a második nibble-ben található. A bebillentett bitek az Enable Voltage mode-ba viszik a belső állapotgépet.
\end{formal}

\subsubsection{Teljesítményfokozat bekapcsolása}
A motor most bekapcsolható az alábbi commandword paranccsal. 
\footnote{Ez a 3. számú állapotátmenet a grafikonon.}
%The drive is now can be switched on. This is transition 3.
%\textbf{ATTENTION!!!! Voltage is about to applied to the BLDC motor!!}

\begin{table}[H]
	\centering
	
	\renewcommand{\arraystretch}{2} % to increase cell height
	\taburulecolor{gray}
	
	\begin{tabu}{|p{1.3cm}|p{1cm}|p{2cm}|p{2cm}|}
		\hline
		COBID & length & data & data
		\\ 		\hline
		0x67F & 8 & 22 40 60 00 & 0f 00 00 00		
		\\		\hline
		0x5FF & 8 & 60 40 60 00 & 00 0 00 00
		\\		\hline
		0x1ff & 2 & 37 06 &
		\\		\hline
	\end{tabu}
	\caption*{Turn on the motor}
\end{table}

A motor ekkor be van kapcsolva és tartónyomatékot tud kifejteni. Ekkor lehetőség van különböző mozgástípusok indítására, a referenciajel lehet pozíció, sebesség vagy gyorsulás is.
%The motor is on, the controller is working and holding the motor in one
%place. Now any motion can be initiated. (according to unit mode...)

%Turning off the motor:
A motor kikapcsolására az alábbi parancs szolgál:

\begin{table}[H]
	\centering
	
	\renewcommand{\arraystretch}{2} % to increase cell height
	\taburulecolor{gray}
	
	\begin{tabu}{|p{1.3cm}|p{1cm}|p{2cm}|p{2cm}|}
		\hline
		COBID & length & data & data
		\\ 		\hline
		0x67F & 8 & 22 40 60 00 & 06 00 00 00		
		\\		\hline
		0x5FF & 8 & 60 40 60 00 & 00 0 00 00
		\\		\hline
		0x1FF & 2 & 31 02 &
		\\		\hline
	\end{tabu}
	\caption*{Turn off the motor}
\end{table}

\begin{formal}
	A kapott TPDO üzenetben látható hogy megváltoztak bizonyos status bitek a kikapcsolt állapotba váltás hatására.
\end{formal}

\subsection{Mozgásprofil beállítása}\label{ptp-motion-point-to-point}

PTP (point to point) mozgást állítunk be:

\begin{itemize}
\itemsep1pt\parskip0pt\parsep0pt
\item
  A mozgás pályája automatikusan számított, a limiteket és paramétereket figyelembe véve: profiled position (PP) típusú
\item
  Körmozgás esetén beállítható modulo mozgás, csak egy forgásirányú mozgás, stb.
%\item
%  a \emph{position demand} value is generated
\end{itemize}

%Trajectory generation includes:
%
%\begin{itemize}
%\itemsep1pt\parskip0pt\parsep0pt
%\item
%  profiled velocity
%\item
%  profiled acceleration/deceleration
%\item
%  selection of motion type and polarity
%\item
%  stopping option
%\end{itemize}

%Thh movements are checked against \textbf{protective limits}.
A mozgás során a védelmi funkciók mindig aktívak. (túláram, beszorulás detektálása)

%\paragraph{Controlword of PP mode:}\label{controlword-of-pp-mode}

%\begin{longtable}[c]{@{}lll@{}}
%\toprule\addlinespace
%Bit & Short & Options
%\\\addlinespace
%\midrule\endhead
%0 & device control & 0: switch on
%\\\addlinespace
%1 & device control & 1: enable voltage
%\\\addlinespace
%2 & device control & 2: NOT quick stop
%\\\addlinespace
%3 & device control & 3: enable operation
%\\\addlinespace
%4 & new setpoint & go to new setpoint
%\\\addlinespace
%5 & change setpoint immediately & or wait to reach current setpoint
%\\\addlinespace
%6 & absolute/relative motion & rel. to the prev. setpoint or not
%\\\addlinespace
%7 & -- Dev. C. dependent
%\\\addlinespace
%8 & halt
%\\\addlinespace
%9..12 & D. C. dep.
%\\\addlinespace
%13 & buffered setpoint
%\\\addlinespace
%14..15 & D. C. dep.
%\\\addlinespace
%\bottomrule
%\end{longtable}

%\paragraph{Statusword of PP mode:}\label{statusword-of-pp-mode}

%\begin{longtable}[c]{@{}lll@{}}
%\toprule\addlinespace
%Bit & Short & Options
%\\\addlinespace
%\midrule\endhead
%0..9 & device control
%\\\addlinespace
%10 & target reached
%\\\addlinespace
%11 & device control
%\\\addlinespace
%12 & new setpoint ACK
%\\\addlinespace
%13 & following error
%\\\addlinespace
%14..15 & D. C. dep.
%\\\addlinespace
%\bottomrule
%\end{longtable}

%\paragraph{PP functional descr.}\label{pp-functional-descr.}

