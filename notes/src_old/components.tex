\section{Eszközök}

A fedélzeti számítógép egyik CAN csatlakozójára kapcsolódik majd a motorvezérlő elektronika. Ám a beágyazott szoftver megírása előtt (agilisan közben.....) egy USB-CAN konverter (UCC) segítségével tudok manuálisan összeállított üzeneteket küldeni a motorvezérlőre.

% Ábra:
\begin{figure}[H]
	\centering
	\begin{tikzpicture}[->,>=stealth']
	\centering
	% Position of FCC 
	% Use previously defined 'state' as layout (see above)
	% use tabular for content to get columns/rows
	% parbox to limit width of the listing
	\node[state,
	yshift=0cm,
	text width=2.5cm] (Matlab) 
	{
		\begin{tabular}{l}
		\textbf{Matlab}\\
		\renewcommand\labelitemi{}
		\parbox{2.5cm}{\begin{itemize}[noitemsep,topsep=0pt,parsep=0pt,partopsep=0pt,leftmargin=-3pt]
			\item CAN üzenetek küldése
			\end{itemize}
		}
		\end{tabular}};
	
	
	% State: UCC
	\node[state,    	% layout (defined above)
	text width=2cm, 	% max text width
	%yshift=2cm, 		% move 2cm in y
	right of=Matlab, 	% Position is to the right of QUERY
	node distance=4cm, 	% distance to FCC
	anchor=center] (UCC) 	% posistion relative to the center of the 'box'
	{%
		\begin{tabular}{l} 	% content
		\textbf{UCC}\\
		\parbox{2cm}{Serial-CAN konverter}
		\end{tabular}
	};
	
	% State: TWI
	\node[state,    	% layout (defined above)
	text width=2.15cm, 	% max text width
	%yshift=2cm, 		% move 2cm in y
	node distance=4cm,
	right of=UCC, 	% Position is to the right of QUERY
	anchor=center]  (TWI) 
	{%
		\begin{tabular}{l} 	% content
		\textbf{Twitter}\\
		\parbox{2.8cm}{motorvezérlő}
		\end{tabular}
	};
	
	\coordinate (CANmid) at ($(UCC)!0.5!(TWI)$);
	
	% STATE NI-CAN
	\node[state,
	node distance=1.5cm,
	anchor=center,
	below of=CANmid,
	text width=4cm] (NI-CAN)
	{%
		\begin{tabular}{l}
		\textbf{CAN bus monitor}\\
		\end{tabular}
	};
	
	% STATE TWI
	\node[state,
	right of=TWI,
	node distance=3.5cm,
	yshift=-1.75cm,
	anchor=center,
	text width=2.15cm] (ENC) 
	{%
		\begin{tabular}{l}
		\textbf{Encoder}\\
		%\parbox{2.8cm}{Dekrementiere Slotzähler}
		\end{tabular}
	};
	
	% STATE BLDC
	\node[state,
	right of=TWI,
	yshift=1.75cm,
	node distance=3.5cm,
	anchor=center] (BLDC) 
	{
		\begin{tabular}{l}
		\textbf{BLDC motor}\\
		%\parbox{4cm}{Mit nächstem \mbox{\textbf{QueryRep}} als "`inventoried"' markieren.}
		\end{tabular}
	};
	
	% draw the paths and and print some Text below/above the graph
	\path 
	(Matlab)	edge[<->]	node[anchor=north,above]{serial}		(UCC)
	(UCC)		edge[<->]	node[anchor=north,above]{CAN} 			(TWI)
	%(CAN)		edge[dashed >] 											(NI-CAN)
	(TWI)  		edge[dashed]                           				(BLDC)
	(BLDC) 		edge[dashed]                           				(ENC)
	(ENC) 		edge												(TWI);
	
	
	\path [<-,draw] (NI-CAN) -- ($ (UCC) !.5! (TWI) $);
	
	\end{tikzpicture}
	\caption{A teszt során használt elrendezés}
\end{figure}

A részegységek:
\begin{itemize}[topsep=-8pt,parsep=0pt,partopsep=-6pt]
	\item Matlab alkalmazás: összeállítja az üzenetet az általunk megadott mezőkkel.
	\item UCC: USB-ről beolvassa azt és CAN-en kiküldi a csomagokat. Ő modellezi a végső hardvert, hiszen a CANopen master funkcionalitást majd az aktuátorokat vezérlő RX-MUX panel kezeli majd.
	\item NI-CAN bus monitor: a CAN buszt figyeli, statisztikát gyűjt, ezen látni a választ (diagnosztikai eszköz). Opcionálisan használható erre a feladatra oszcilloszkóp vagy logikai analizátor is.
	\item 	Az Elmo Gold Twitter egy olyan hajtásszabályzó elektronika, amiben integráltan megtalálható:
	\begin{itemize}
		\item az I/O és kommunikációs egység, ennek a CANopen fukncióit használjuk,
		\item a szabályzó, illetve védelmi és biztonsági funkciókat is ellátó processzor egység,
		\item a teljesítményfokozat, ami a szabályzó beavatkozó jelét a motorra küldi,
		\item és bemenet a visszacsatoláshoz, pl. inkrementális encoderhez.
	\end{itemize}
\end{itemize}


\begin{figure}[H]
	\centering
	\includegraphics[width=15cm]{figures/hw/comm2}
	\caption{Teszt összeállítás}
	\label{fig-DD}
\end{figure}

%
%\begin{scriptsize}
%    \textcolor{red}{Értem én a próbálkozást ez mögött a kép mögött, de az egy mérnököknek készülő dokumentum, nem pedig youtube videó 10 éveseknek... és nem is reklám. Azért van a képhez jegyzet, hogy ott fel lehessen sorolni, hogy mi van a képen és miért. Illetve elvadult esetben lehet a komponensekről külön is képet csinálni, de az más néha sok.
%    }
%\end{scriptsize}


