\section{Motiváció}


\begin{scriptsize}
\textcolor{red}{
A motiváció mint olyan megegyezik a bevezetővel, legalábbis én oda tenném. Így olyan mintha két független bevezetőd lenne, mindkettőben közölsz infókat, de nem eleget ahhoz, hogy meg lehessen pontosan érteni, hogy mit is szeretnél.
}
\end{scriptsize}


Távirányítós modellrepülőkön a pilóta távirányítójával számos beavatkozót tud kezelni. Joystickek, karok, csúszkák állnak a rendelkezésére, melyeknek a mozgatása pl. egy-egy szárnyfelületet térít ki, jellemzően RC szervókat mozgatva.
%Az eredmény látható, a gép mozgása a levegőben megváltozik, a pilóta pedig különféle manővereket hajthat végre.
Az ilyen szervók már nagyon olcsón elérhetők és kiválóan alkalmasak könnyű, egyszerű gépek építésére.
%Egyes szervók nagyobb igénybevételre, teljesítményre is képesek, miközben vezérlésük megegyezik gyengébb társaival. 
%(Kép a Graupnerről)
Alkalmazásuk sem bonyolult, a távirányítóhoz tartozó vevőegységen keresztül PWM jelet kapnak, ez megadja a szöghelyzetüket. A mozgásról viszont semmi információnk nincs, alapvetően nem kapjuk vissza a tényleges kitérést, a felvett áramot vagy a mozgás sebességét. A pozíciószabályzás a szervó belsejében történik, a külvilágtól elzárva.

\vspace{10pt}

A H2020 FLEXOP projekt keretében épül egy pilóta nélküli demonstrátor repülőgép, amin egy aerodinamikai jelenség, a flutter aktív csillapítását kutatják. Bizonyos sebesség felett ugyanis a szárnyakon egyre erősödő, oszcilláló mozgás alakul ki a repülőgép szárnyának hajlékonysága miatt. Ennek frekvenciája kb. \SI{10}{\hertz}. A fedélzeten található elektronika ennek kialakulását méri és ezalapján beavatkozik. A beavatkozó jel a zavaráshoz képest nagyobb sávszélességű lesz, ezt pedig az aktuátornak követnie kell. Az RC szervók erre nem képesek, szükség van a kb. \SI{90}{\hertz}-es határfrekvenciájú BLDC alapú szervóhajtásra. Több ok miatt esett tehát erre a választás:


\begin{scriptsize}
\textcolor{red}{
    A 90Hz-et a cucc nem tud, és erősen függ, hogy milyen alkalmazásban próbálod használni, így ezt ebben a formában nem írnám le. Elég az, hogy egy ilyen megoldás képes a megfelelő tulajdonságok biztosítására, a hagyományos szervók pedig nem.
    Továbbá adott sávszélesség mellett a maximális nyomaték is fontos szempont, és azt sem tudják az RC-s szarok.
}
\end{scriptsize}



\begin{itemize}
	\item Működése nem helyettesíthető. A többi szárnyfelület szervóinak meghibásodása esetén a megmaradt működő egységek át tudják venni a hibásak feladatát és a géppel le lehet szállni. A BLDC-s szervóhajtás meghibásodása katasztrofális lehet.
\begin{scriptsize}
    \textcolor{red}{Ez biztosan olyan infó ami fontos a mostani munkád szempontjából? Szerintem semmi köze a CAN es kommunikációnak a redundancióhoz.
}
\end{scriptsize}


	\item Lényegesen szigorúbbak a követelmények pl. a beállási vagy felfutási időkre, hiszen az alapjel sokkal gyorsabban (nagyobb frekvenciával) változik, mint egy pilóta parancsai a távirányítóról vagy az autopilóta által meghatározott refrenciajel.

\begin{scriptsize}
\textcolor{red}{Mihez képest szigorúbbak a követelmények? Cserélj szórendet az érthetőség kedvéért. Ehez tartozó alapjelet is az autó pilóta fogja generálni nem? 
}
\end{scriptsize}


	\item A flutter szabályzónak (a magasabb szintű szabályzási körnek) pontosan tudnia kell hogy az alapjel-követés megvalósul-e.  
\end{itemize}


A Gold Twitter motorvezérlő elektronika használatával a fenti követelményeket kielégíthetjük, számos paramétert mérhetünk, monitorozhatunk, a mozgás tulajdonságait beállíthatjuk (sebesség, pozíció, gyorsulások), az alapjelkövetést ellenőrizni tudjuk. Probléma esetén hibajegyeket küld a Twitter, ezért a diagnosztika is könnyebb.

\begin{scriptsize}
\textcolor{red}{Valami olyasmire kellene kihegyezni, hogy egy megfelelő BLCD motor és hozzá tartozó adott kiegészítők segítségével, a fent leírt követelmények megvalósíthatóak. De ez még mindig az Önlabod bevezetőjének kellene lennie, szóval onnan érdemes átvenni.
}
\end{scriptsize}



\textbf{A munkám az volt, hogy biztosítsam azt, hogy a flutter controller és a motorvezérlő közötti kapcsolatot megteremtsem és olyan adatokat küldjek vissza a fedélzeti számítógépre amiket logolni szükséges.}

