\begin{figure}[!h]
	\centering
	\begin{tikzpicture}[->,>=stealth']
	\centering
	% Position of FCC 
	% Use previously defined 'state' as layout (see above)
	% use tabular for content to get columns/rows
	% parbox to limit width of the listing
	\node[state,
	yshift=0cm,
	text width=3cm] (FCC) 
	{
		\begin{tabular}{l}
		\textbf{Raspberry Pi}\\
		\renewcommand\labelitemi{}
		\parbox{3cm}{\begin{itemize}[noitemsep,topsep=0pt,parsep=0pt,partopsep=0pt,leftmargin=-3pt]
			\item flutter szabályzó
			\item adatmentés (log)
			\item autopilóta
			\end{itemize}
		}\\[2em]
		\textbf{flightHAT}\\
		\renewcommand\labelitemi{}
		\parbox{3cm}{\begin{itemize}[noitemsep,topsep=0pt,parsep=0pt,partopsep=0pt,leftmargin=-3pt]
			\item hardver interface
			\end{itemize}
		}
		\end{tabular}};
	
	% STATE Sensor
	\node[state,
	below of=FCC,
	node distance=3cm,
	anchor=center,
	text width=2.5cm] (Sensor) 
	{%
		\begin{tabular}{c}
		\parbox{2.2cm}{\textbf{Szenzorok, földi állomás}}
		\end{tabular}
	};
	
	% State: RX-MUX
	\node[state,    	% layout (defined above)
	text width=2cm, 	% max text width
	%yshift=2cm, 		% move 2cm in y
	right of=FCC, 	% Position is to the right of QUERY
	node distance=5cm, 	% distance to FCC
	anchor=center] (RX) 	% posistion relative to the center of the 'box'
	{%
		\begin{tabular}{l} 	% content
		\textbf{RX-MUX}\\
		%\parbox{2cm}{Bestätigen mit $RN_{16}$}
		\end{tabular}
	};
	
	% STATE RC
	\node[state,
	below of=RX,
	node distance=2cm,
	anchor=center,
	text width=4cm] (RC) 
	{%
		\begin{tabular}{l}
		\textbf{RC szervók, RC vevők}\\
		%\parbox{2.8cm}{Dekrementiere Slotzähler}
		\end{tabular}
	};
	
	
	% State: TWI with different content
	\node[state,    	% layout (defined above)
	text width=3cm, 	% max text width
	%yshift=2cm, 		% move 2cm in y
	right of=RX, 	% Position is to the right of QUERY
	node distance=5cm, 	% distance to FCC
	anchor=center] (TWI) 	% posistion relative to the center of the 'box'
	{%
		\begin{tabular}{l} 	% content
		\textbf{Twitter}\\
		\parbox{2.8cm}{motorvezérlő}
		\end{tabular}
	};
	
	% STATE ENC
	\node[state,
	right of=TWI,
	node distance=3.5cm,
	yshift=-1.75cm,
	anchor=center,
	text width=3cm] (ENC) 
	{%
		\begin{tabular}{l}
		\textbf{Encoder}\\
		%\parbox{2.8cm}{Dekrementiere Slotzähler}
		\end{tabular}
	};
	
	% STATE BLDC
	\node[state,
	right of=TWI,
	yshift=1.75cm,
	node distance=3.5cm,
	anchor=center] (BLDC) 
	{
		\begin{tabular}{l}
		\textbf{BLDC motor}\\
		%\parbox{4cm}{Mit nächstem \mbox{\textbf{QueryRep}} als "`inventoried"' markieren.}
		\end{tabular}
	};
	
	% draw the paths and and print some Text below/above the graph
	\path 
	(FCC) 	
	edge[<->] %[bend left=20]  
	node[anchor=south,above=5pt]
	{\begin{tabular}{l}
		\parbox{2cm}{autopilóta referenciajelei}
		\end{tabular}
	}	(RX)
	%node[anchor=north,below]{$RN_{16}$} 	(RX)
	(FCC)		edge[<->]  											(Sensor)
	(RX)		edge[<->]  											(RC)
	(RX)		edge[<->]  	node[anchor=north,above]{CANopen} 		(TWI)
	(TWI)  	edge[dashed]                               				(BLDC)
	(BLDC) 	edge[dashed]                            				(ENC)
	(ENC) 		edge												(TWI);
	
	\end{tikzpicture}
	\caption{A repülőgépen található elrendezés}
\end{figure}