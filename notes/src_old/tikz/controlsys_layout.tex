\begin{tikzpicture}[->,>=stealth']
\centering
% Position of FCC 
% Use previously defined 'state' as layout (see above)
% use tabular for content to get columns/rows
% parbox to limit width of the listing
\node[state,
yshift=0cm,
text width=3cm] (PC) 
{
	\begin{tabular}{l} 	% content
		\textbf{Számítógép}\\
		\parbox{2.8cm}{	Elmo Application Studio}
	\end{tabular}
};


% State: TWI with different content
\node[state,    	% layout (defined above)
text width=3cm, 	% max text width
%yshift=2cm, 		% move 2cm in y
right of=PC, 	% Position is to the right of QUERY
node distance=5cm, 	% distance to FCC
anchor=center] (TWI) 	% posistion relative to the center of the 'box'
{%
	\begin{tabular}{l} 	% content
	\textbf{Twitter}\\
	\parbox{2.8cm}{motorvezérlő}
	\end{tabular}
};

% STATE ENC
\node[state,
right of=TWI,
node distance=3.5cm,
yshift=-1.75cm,
anchor=center,
text width=3cm] (ENC) 
{%
	\begin{tabular}{l}
	\textbf{Encoder}\\
	%\parbox{2.8cm}{Dekrementiere Slotzähler}
	\end{tabular}
};

% STATE BLDC
\node[state,
right of=TWI,
yshift=1.75cm,
node distance=3.5cm,
anchor=center] (BLDC) 
{
	\begin{tabular}{l}
	\textbf{BLDC motor}\\
	%\parbox{4cm}{Mit nächstem \mbox{\textbf{QueryRep}} als "`inventoried"' markieren.}
	\end{tabular}
};

% draw the paths and and print some Text below/above the graph
\path
(PC)		edge[<->]  	node[anchor=north,above]{Serial} 		(TWI)
(TWI)  	edge[dashed]                               				(BLDC)
(BLDC) 	edge[dashed]                            				(ENC)
(ENC) 		edge												(TWI);

\end{tikzpicture}